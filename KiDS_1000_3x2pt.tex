%                                                                 aa.dem
% AA vers. 8.2, LaTeX class for Astronomy & Astrophysics
% demonstration file
%                                                       (c) EDP Sciences
%-----------------------------------------------------------------------
%
%\documentclass[referee]{aa} % for a referee version
%\documentclass[onecolumn]{aa} % for a paper on 1 column  
%\documentclass[longauth]{aa} % for the long lists of affiliations 
%\documentclass[rnote]{aa} % for the research notes
%\documentclass[letter]{aa} % for the letters 
%\documentclass[bibyear]{aa} % if the references are not structured 
% according to the author-year natbib style

%
\documentclass[letter]{aa}  
%
%

\usepackage{graphicx}
%%%%%%%%%%%%%%%%%%%%%%%%%%%%%%%%%%%%%%%%
\usepackage{txfonts}
%%%%%%%%%%%%%%%%%%%%%%%%%%%%%%%%%%%%%%%%
\usepackage{hyperref}
\pdfoutput=1
\usepackage{amsfonts}
\usepackage{amsmath}
\usepackage{amssymb}
\usepackage{xcolor}
\usepackage{enumitem}
\usepackage{graphicx}
\usepackage{gensymb}
\usepackage{fancyhdr}
\usepackage{times}
\usepackage{ulem}
%\usepackage{caption}
\usepackage{subcaption} 
\usepackage{float} 
\usepackage{color}
\usepackage{booktabs}
 

\definecolor{purple}{RGB}{76, 0,153}
\newcommand{\ch}[1]{{\color{purple}{#1}}}
\newcommand{\bg}[1]{\textcolor{orange}{#1}}

\newcommand{\dd}{{\rm d}}
\newcommand{\br}[1]{\left( #1 \right)}
\newcommand{\bc}[1]{\left\{ #1 \right\}}
\newcommand{\eqa}[1]{\begin{align}   #1 \end{align}}

\newcommand{\bb}[1]{\left[ #1 \right]}
\newcommand{\vek}[1]{\mbox{\boldmath $#1$}}


\newcommand{\be}{\begin{equation}}  \newcommand{\ee}{\end{equation}}


\begin{document} 

   \title{KiDS-1000 Cosmology: Multi-probe weak gravitational lensing and spectroscopic galaxy clustering constraints}

   \author{Catherine Heymans \inst{1,2}\thanks{Catherine Heymans: heymans@roe.ac.uk} 
   \and Tilman Tr\"oster\inst{1}\thanks{Tilman Tr\"oster: ttr@roe.ac.uk} 
   \and Marika Asgari\inst{1} 
   \and Chris Blake\inst{3}
   \and Hendrik Hildebrandt\inst{2}
   \and Benjamin Joachimi\inst{4}
   \and Chieh-An Lin\inst{1}
   \and Ariel Sanchez\inst{5}
   \and Tilman Tr\"oster\inst{1}
   \and Jan Luca Van den Busch\inst{2}
   \and Angus Wright\inst{2}
   \and the KiDS Collaboration
          }
\institute{Institute for Astronomy, University of Edinburgh, Royal Observatory, Blackford Hill, Edinburgh, EH9 3HJ, UK 
   \and
   Ruhr-Universität Bochum, Astronomisches Institut, German Centre for Cosmological Lensing (GCCL), Universitätsstr. 150, 44801, Bochum, Germany
   \and 
   Centre for Astrophysics \& Supercomputing, Swinburne University of Technology, P.O. Box 218, Hawthorn, VIC 3122, Australia
   \and
   University College London, Gower Street, London WC1E 6BT, UK
   \and
   Max-Planck-Institut f\"ur extraterrestrische Physik, Postfach 1312, Giessenbachstrasse 1, D-85741 Garching, Germany
   %Department of Astrophysical Sciences, Princeton University, 4 Ivy Lane, Princeton, NJ 08544, USA
     % \and
   %Department of Physics, University of Oxford, Denys Wilkinson Building, Keble Road, Oxford OX1 3RH, U.K.
    }
   % Leiden Observatory, Leiden University, Niels Bohrweg 2, 2333 CA Leiden, the Netherlands
 %  \date{Received September 15, 1996; accepted March 16, 1997}

% 5 {} token are mandatory
 
  \abstract{TBD}

 \keywords{gravitational lensing: weak, methods: data analysis, methods: statistical, surveys, cosmology: observations}

   \titlerunning{KiDS-1000: 3x2pt}
   \authorrunning{Heymans, Tr\"oster \& the KiDS Collaboration et al.}
   \maketitle
%
%________________________________________________________________
\section{Introduction}
\label{sec:intro}

Observations of the cosmic microwave background (CMB) have delivered high precision
constraints for the cosmological parameters of the flat, cold dark
matter and cosmological constant model of the Universe
\citep[$\Lambda$CDM,][]{planck/etal:2018}.  With only six free
parameters, this flat $\Lambda$CDM model provides an exquisite fit to observations of
the anisotropies in the CMB.    The same model predicts a range of
different observables in the present day Universe, including the
distribution of, and gravitational lensing by, large-scale
structures \citep{peebles/1980,bartelmann/schneider:2001,eisenstein/etal:2005}, and the rate
of cosmic expansion \citep{weinberg/1972}.  In most cases, there
is agreement between the cosmological parameters of the flat $\Lambda$CDM model
constrained at the CMB epoch, and through a variety of
lower redshift probes \citep[see the discussion in][and references
therein]{planck/etal:2018}.   Recent increases in the statistical
precision of the lower redshift probes has,
however, revealed some statistically significant differences.  Most
notably a $4.4\sigma$ difference in the value of the Hubble constant
$H_0$ has been reported in \citet{riess/etal:2019}.  If this difference can not be
attributed to systematic errors in either, or both, experiment, this result
suggests that the flat $\Lambda$CDM model is incomplete.

Many
extensions have been proposed to reconcile the observed differences between
high and low redshift probes \citep[see for
example][]{poulin/etal:2018,divalentino/etal:2020}.  Each, however, require
an additional component
to the cosmological model that moves even further away from the
standard model of particle physics, a model that already struggles to motivate
the existence of cold dark matter and a cosmological constant.   In these closing phases of the `Stage-III'
surveys, in the run up to the`full-sky' imaging and spectroscopic
cosmology surveys of the 2020's \citep[Euclid, VRO/LSST and DESI,][]{laureijs/etal:2011,lsst/etal:2009,DESI/etal:2016}, the highest
priority is to develop a keen understanding, and mitigation, of systematic errors.

In this analysis we present a multi-probe `same-sky' analysis of the
evolution of large-scale structures, using imaging and spectroscopic surveys.
Our first observable is the weak gravitational lensing of background
galaxies by foreground large scale
structures, commonly referred to as `{\it cosmic shear}'.    Our second
observable is the {\it anisotropic clustering of galaxies} within these
large-scale structures, combining measurements of both redshift-space
distortions and the redshift-dependent scale of the baryon acoustic
oscillation.   Our third observable is the weak gravitational lensing of background
galaxies by foreground galaxies, commonly referrred to as
`{\it galaxy-galaxy lensing}'.   As these three sets of two point
statistics are analysed simultaneously, this combination of probes
is commonly referred to as a `$3\times 2{\rm pt}$' analysis. 

Each observable in our multi-probe analysis is subject to systematic
uncertainties.  For a cosmic shear analysis, the observable is a
combination of the true cosmological signal with a low-level signal
arising from the intrinsic alignment of galaxies and potential residual
correlations in the data induced by the atmosphere, telescope and
camera.   The whole signal can be scaled by both shear
 and photometric redshift measurement calibration errors
 \citep[see][and references therein]{mandelbaum:2018}.   For a galaxy
   clustering analysis, the observable is the true
   cosmological signal modulated by an uncertain, non-linear and
   evolving, galaxy bias function.  It is this function that maps how
   the galaxies trace the
   underlying total matter distribution \citep[see][and references
   therein]{desjacques/etal:2018}.   Finally the galaxy-galaxy
   lensing analysis is subject to the systematics that impact both the
   cosmic shear and clustering analyses.

   When analysising these
   observables in combination
   the different astrophysical and systematic dependencies allow for some degree of
   self-calibration \citep{bernstein/jain:2004, hu/jain:2004,
     bernstein:2009,joachimi/bridle:2010}.  `Same-sky'
   surveys, overlapping imaging for weak lensing observables, with
   spectroscopy for anisotropic galaxy clustering observables,
   also allows for their cross-correlation.  Such a survey design therefore
   presents a robust
   cosmological tool that can calibrate and mitigate systematic and astrophysical
   uncertainties through a series of nuisance parameters.   In
   addition to enhanced systematic control, this combination of probes
   breaks cosmological parameter degeneracies from each individual
   probe, decreasing the
   uncertainty on the recovered dark energy equation of state
   parameter \citep{hu/jain:2004,gaztanaga/etal:2012}.

   Three variants of a joint `$3\times 2{\rm pt}$' analysis have been
   conducted to date.  \citet{vanuitert/etal:2018} present a joint power-spectrum
   analysis of the Kilo-Degree Survey \citep[KiDS][]{kuijken/etal:2015} with
   the Galaxies And Mass Assembly survey
   \citep[GAMA][]{liske/etal:2015}, incorporating projected
   angular clustering measurements.   \citet{joudaki/etal:2018}
   present a joint analysis of KiDS with the
   2-degree Field Lensing Survey \citep[2dFLenS][]{blake/etal:2016}
   and the overlapping area in the Baryon Oscillation Spectroscopic Survey \citep[BOSS][]{alam/etal:2015}, incorporating
   redshift-space clustering measurements.  \citet{abbott/etal:2018}
   present a joint real-space lensing-clustering analysis of the Dark
   Energy Survey \citep[DES][]{drlicawagner/etal:2018}, using a high-quality
   photometric redshift sample of luminous red galaxies for their projected
   angular clustering measurements.  In all three cases an effective
   linear galaxy bias model was adopted.

   In this analysis we enhance and build upon the advances of previous `$3\times
   2{\rm pt}$' studies.   We analyse the most recent KiDS data release \citep[KiDS-1000][]{kuijken/etal:2019}, more than doubling the
   survey area from previous KiDS analyses.   We utilize the full-BOSS
   area and the `full-shape' anisotropic clustering measurements of \citet{sanchez/etal:2017},
   incorporating information from both redshift-space distortions
   and the baryon acoustic oscillation as our galaxy clustering probe.   We adopt a non-linear
   galaxy bias model, derived from renormalised perturbation theory
   \citep{crocce/scoccimarro:2006, chan/etal:2012}.


   











   
\section{Data}
\label{sec:data}

\subsection{Surveys:  KiDS, BOSS and 2dFLenS}
\label{sec:surveys}

The Kilo-Degree Survey \citep[KiDS,][]{dejong/etal:2013}, spans 1350 sq degrees split into two fields, one
equatorial and one southern.    Matched-depth imaging in nine bands spans the optical,
$urgi$, through to the near-infra-red, $ZYJHK_s$, where the
near-infra-red imaging was taken as part of the KiDS partner survey
VIKING \citep[the VISTA Kilo-degree INfrared Galaxy
survey,][]{edge/etal:2013}.  High quality seeing was
routinely allocated to the primary KiDS $r$-band VST-OmegaCAM observations, resulting in a
mean $r$-band seeing of 0.7 arcseconds, with a maximum of 0.8
arcseconds.  This combination of full-area spatial and wavelength
resolution over a thousand square degrees,
provides a unique weak lensing survey that allows for enhanced
control of systematic errors \citep{giblin/etal:inprep, hildebrandt/etal:inprep}.
This analysis uses data from the fourth KiDS
data release of 1006 square degrees of imaging, (hence the name KiDS-1000), which has an effective
area, after masking, of 777 square degrees.  KiDS is a public survey from the European Southern
Observatory, with data products freely accessible through the ESO
archive\footnote{KiDS data access: \href{http://kids.strw.leidenuniv.nl/DR4}{kids.strw.leidenuniv.nl/DR4}}.   

The Baryon Oscillation Spectroscopic Survey
\citep[BOSS,][]{alam/etal:2015}, spans an effective area of 9329 square
degrees, with spectroscopic redshifts for 1.2 million luminous red
galaxies (LRG) in the redshift range $0.2<z<0.9$.   A range of
different statistical analyses of the clustering of BOSS galaxies have been used in combination with CMB
measurements, to set tight contraints on extensions to the standard
flat $\Lambda$CDM model \citep[see][and references
therein]{alam/etal:2017}.   We adopt the anisotropic clustering
measurements of \citet{sanchez/etal:2017} in this multi-probe analysis.
BOSS only overlaps with the equatorial stripe
of the KiDS survey, with 409 square degrees of the BOSS survey lying within
the KiDS-1000 footprint.  BOSS galaxies in this overlapping region are used as lenses in
our galaxy-galaxy lensing analysis, with an effective lens number density of 0.031
galaxies per square arcmin.  BOSS is a public survey from the third Sloan
Digital Sky Survey\footnote{BOSS data access: \href{https://data.sdss.org/sas/dr12/boss/lss/}{data.sdss.org/sas/dr12/boss/lss/}}.   

The two-degree Field Lensing Survey
\citep[2dFLenS,][]{blake/etal:2016}, spans 731 square degrees, with
spectroscopic redshifts for 70,000 galaxies out to $z<0.9$.   This
galaxy redshift survey from the Anglo-Australian Telescope was designed
to target areas already mapped by weak lensing surveys to facilitate `same-sky'
lensing-clustering analyses
\citep{johnson/etal:2017,amon/etal:2018,joudaki/etal:2018}.
We use data from the 2dFLenS LRG sample that was targetted to match
the BOSS-LRG selection, but with sparser sampling.  2dFLenS
thus provides an additional sample of BOSS-like galaxies in the KiDS
southern stripe where there is 425 square degrees of overlap within
the KiDS-1000 footprint.  2dFLenS galaxies in this overlapping region are used as lenses in
our galaxy-galaxy lensing analysis, with an effective lens number density of 0.012
galaxies per square arcmin.  2dFLenS is a public survey\footnote{2dFLenS data
  access: \href{http://2dflens.swin.edu.au/data.html}{2dflens.swin.edu.au/data.html}}.   




\subsection{Cosmic Shear}
\label{sec:cosmic_shear}
The KiDS-1000 cosmic shear power spectra from \citet{asgari/etal:inprep} is shown in Figure~\ref{fig:Pkk}.

\begin{figure*}
        \includegraphics[width=\textwidth]{Data_Plots/Pkk/Pkk_K1000_2Dbins_v2_goldclasses_Flag_SOM_Fid.png}
        \caption{KiDS-1000 cosmic shear power spectra:  Tomographic
          band powers comparing the E-modes (upper left block) with the best-fit
          cosmological model from our combined multi-probe analysis
          \ch{TO DO}.  The tomographic
        bin combination is indicated in the upper right corner of each
      sub-panel.  The null-test B-modes (lower right block), are
      consistent with zero for both the full data vector, and each
     bin combination individually.}
        \label{fig:Pkk}
\end{figure*}

\subsection{Galaxy-Galaxy Lensing}
\label{sec:GGL}
Summary of Blake et al?


In Figure~\ref{fig:Pgk} we present the KiDS-1000 galaxy-galaxy lensing
power spectra, around lenses from the BOSS and 2dFLenS surveys.

\begin{figure*}
        \includegraphics[width=\textwidth]{Data_Plots/Pgk/Pgk_K1000_2Dbins_v2_goldclasses_Flag_SOM_Fid.png}
        \caption{KiDS-1000 galaxy-galaxy lensing power spectra:
          Tomographic band powers comparing the E-modes (left block)
          with the best-fit
          cosmological model from our combined multi-probe analysis
          \ch{TO DO}.  The tomographic 
        bin combination of BOSS and 2dFLenS lenses (L) with KiDS-1000
        sources (S), is indicated in the upper right corner of each
        sub-panel.  Data within grey-regions are not included in the cosmological analysis.
        The null-test B-modes (right block), are
      consistent with zero for both the full data vector, and each
     bin combination individually \ch{TO DO}.}
        \label{fig:Pgk}
\end{figure*}

\subsection{Anisotropic Galaxy Clustering}
\label{sec:clustering}
Summary of \citet{sanchez/etal:2017}

In Figure~\ref{fig:wedges} we present the BOSS-DR12 anisotropic
clustering wedges.
\begin{figure*}
        \includegraphics[width=\textwidth]{Data_Plots/clustering_wedges/BOSS_Sanchez_wedges.png}
        \caption{BOSS-DR12 anisotropic clustering from \citet{sanchez/etal:2017}:
          The transverse (pink), intermediate (blue) and parrallel
          (black) clustering wedges in two redshift bins, compared 
          with the best-fit
          cosmological model from our combined multi-probe analysis
          \ch{TO DO}.}
        \label{fig:wedges}
\end{figure*}

\subsection{Covariance}
\label{sec:Cov}
Summary of \citet{joachimi/etal:inprep}


\section{Results}
\label{sec:results}
We present our multi-probe constraints on the cosmological parameters of the flat $\Lambda$CDM model in Fig.~\ref{fig:cosmology-params}, showing the marginalised posterior distributions for $\sigma_8$, $\Omega_{\rm m}$ and $h$, where the BOSS galaxy clustering constraints (GC: shown blue), break the $\sigma_8-\Omega_{\rm m}$ degeneracy in the KiDS-1000 cosmic shear constraints (CS: shown pink), resulting in tight constraints on $\sigma_8$ in the combined \tttp analysis (shown red). 
Reporting the MAP values with PJ-HPD credible intervals for the parameters that we are most sensitive to, we find 
\preliminary{\eqa{
\sigma_8 &= 0.76^{+0.021}_{-0.023} \\ \nonumber
\Omega_{\rm m} &= 0.306^{+0.011}_{-0.014} \\ \nonumber
S_8 &= 0.769^{+0.018}_{-0.015} \, .
}}
Our constraints can be compared to the marginalised posterior distributions from Planck (shown green in Fig.~\ref{fig:cosmology-params}), finding consistency between the marginalised constraints on $\Omega_{\rm m}$ and $h$, but an offset in $\sigma_8$,  which we discuss in detail in Sect.~\ref{sec:planck_comp}.

Tabulated constraints for the full set of cosmological parameters are presented in Appendix~\ref{app:parameter-constraints}, quoting our fiducial MAP with PJ-HPD credible intervals, along with the marginal posterior mode with M-HPD credible intervals. 
We note that the quoted marginal mode constraint on $S_8$ is \preliminary{$0.2\sigma$} lower than the MAP for this parameter. 
As discussed in \citet{joachimi/etal:inprep}, this marginal mode estimate is known to yield systematically low values of $S_8$ in mock data analyses, as can be seen in Fig.~\ref{fig:S8comp}, which compares the joint posterior constraints (solid) with the marginal posterior constraints (dashed).  

\begin{figure}
	\begin{center}
		\includegraphics[width=\columnwidth]{Parameter_Plots/cosmology/omegam_sigma8_h_blind_C}
		\caption{Marginal multi-probe constraints on the flat $\Lambda$CDM cosmological model, for the matter fluctuation amplitude parameter, $\sigma_8$, the matter density parameter $\Omega_{\rm m}$, and the Hubble parameter, $h$.  The BOSS galaxy clustering constraints (GC: shown blue), can be compared to the KiDS-1000 cosmic shear constraints (CS: shown pink), the combined $3\times2{\rm pt}$ analysis (shown red), and CMB constraints from \citet{planck/etal:2018}.}
		\label{fig:cosmology-params}
	\end{center}
\end{figure}

We find good agreement between the different probe combinations and single-probe $S_8$ constraints, demonstrating internal consistency between the different cosmological probes, in Fig.~\ref{fig:S8comp}.  
As forecast by \citet{joachimi/etal:inprep}, the addition of the galaxy-galaxy lensing observable adds very little constraining power, with similar results found for the full \tttp analysis and the combined cosmic shear and clustering analysis. 
This is a result of the significant area of BOSS in comparison to KiDS-1000, and the fact that our lack of an accurate galaxy bias model on the deeply non-linear scales that weak lensing probes,
prohibits the inclusion of large sections of our galaxy-galaxy lensing data vector, shown in Fig.~\ref{fig:Pgk}.  
The addition of the galaxy-galaxy lensing does however serve to moderately tighten constraints on the amplitude of the intrinsic alignment model $A_{\rm IA}$,  as seen in Fig.~\ref{fig:cosmology-params-all}. 

Fig.~\ref{fig:S8comp} also demonstrates the good agreement between our constraints and weak lensing results from the literature, comparing to cosmic-shear only results from the Hyper Suprime-Cam Strategic Program \citep[HSC,][]{hikage/etal:2019,hamana/etal:2020}, DES \citep{troxel/etal:2018} and previous KiDS analysis \citep[KV450][]{hildebrandt/etal:2020}, in addition to the previous KV450-BOSS `$2\times2$pt' analysis of \citet{troester/etal:2020} and the DES Y1 \tttp analysis from \citet{abbott/etal:2018}.   We refer the reader to \citet{asgari/etal:inprep} for a detailed discussion and comparison of our cosmic-shear results, and Sect.~\ref{sec:WL_comp} for a more detailed comparison with \tttp results in the literature.

\begin{figure}
	\begin{center}
		\includegraphics[width=\columnwidth]{Parameter_Plots/systematics/S8_comparison_blindC}
		\caption{Constraints on structure growth parameter $S_{8} = \sigma_8 \sqrt{\Omega_{\rm m}/0.3}$ for different probe combinations: $3\times2$pt, cosmic shear with galaxy-galaxy lensing (CS+GGL), and cosmic shear with galaxy clustering (CS+GC), along with the cosmic shear (CS), and galaxy clustering (GC), single probe analyses.   Our fiducial and preferred MAP with PJ-HPD credible interval (solid) can be compared to the standard, but biased, marginal posterior mode with M-HPD credible intervals (dashed).    Our results can also be compared to weak lensing measurements from the literature, which typically quote the mean of the marginal posterior mode with tail credible intervals (dotted). 
		\ch{TO DO:  Add CS+GC to this figure}
		\label{fig:S8comp}}
	\end{center}
\end{figure}

Fig.~\ref{fig:cosmology-params-all} displays the marginal posterior distributions for an extended set of cosmological parameters.  
We find that the constraint on the linear galaxy bias,  $b_1$, in each redshift bin (lower two rows), is more than halved with the addition of the weak lensing data. 
This constraint does not arise, however, from the sensitivity of the galaxy-galaxy lensing observable to galaxy bias (shown to be relatively weak in the CS+GGL contours). 
Instead, in this analysis, it is a result of the degeneracy breaking in the $\sigma_8-\Omega_{\rm m}$ plane, tightening constraints on $\sigma_8$ which, for galaxy clustering, is degenerate with galaxy bias. 
The improved constraints on galaxy bias do not, however, fold through to improved constraints on $h$, which the weak lensing data adds very little information to. 

\ch{For our primary cosmological parameter, $S_8$, our constraints are uninformed by our choice of priors.    This statement cannot be made for the other $\Lambda$CDM parameters, however, as shown in Fig.~\ref{fig:cosmology-params-all}.   The most informative prior that we have introduced to our \tttp analysis is on the spectral index, $n_{\rm s}$.  As noted by \citet{troester/etal:2020}, the BOSS galaxy clustering constraints favour a low value for $n_{\rm s}$, where they find $n_{\rm s} = 0.815 \pm 0.085$. 
From the \citet{troester/etal:2020} sensitivity analysis to the adopted maximum clustering scale, we observe that this preference appears to be driven by the amplitude of the large scale clustering signal with $s > 100 \, h^{-1}\, {\rm Mpc}$.  We note that spurious excess power in this regime could plausibly arise from variations in the stellar density impacting the BOSS galaxy selection function \citep{ross/etal:2017}.  Our choice to impose a theoretically motivated informative prior for $n_{\rm s}$, as listed in Table~\ref{tab:priors}, helps to negate this potential systematic effect without degrading the overall goodness of fit to the galaxy clustering measurements.  Our prior choice is certainly no more informative than the $n_{\rm s}$ priors that are typically used in weak lensing and clustering analyses \citep[see for example][]{abbott/etal:2018}. 
We recognise, however, that this well-motivated prior choice acts to improve the BOSS-only error on $\Omega_{\rm m}$ by roughly a third, and decrease the BOSS-only best-fitting value for $\Omega_{\rm m}$ and $h$ by $\sim 0.5\sigma$.  With $<10\%$ differences on the constraints on $S_8$ and $h$, however, and only a $\sim 0.1\sigma$ difference in the BOSS-only best-fitting value for $S_8$, which is consistent with the typical variation between different MCMC analyses, we conclude that our prior choice does not impact on our primary $S_8$ constraints.   With the informative or uninformative $n_s$ prior, our constraints on $h$ remain consistent with the Hubble parameter constraints from both \citet{planck/etal:2018} and \citet{riess/etal:2019}.}

\begin{figure*}
	\begin{center}
		\includegraphics[width=\textwidth]{Parameter_Plots/cosmology/omegam_sigma8_s8_ns_h_a_baryon_a_ia_b1l_b1h_blind_C}
		\caption{Marginalised posterior distributions for an extended set of cosmological parameters covering the matter density parameter $\Omega_{\rm m}$, the matter fluctuation amplitude parameter, $\sigma_8$, the structure growth parameter $S_8$, the spectral index $n_s$, the Hubble parameter, $h$, the baryon feedback amplitude parameter, $A_{\rm baryon}$, the intrinsic alignment amplitude, $A_{\rm IA}$, and the linear bias parameters for the low and high BOSS redshift bins, $b_1$.   The KiDS-1000 cosmic shear results (CS, pink), can be compared to the BOSS galaxy clustering results (GC, blue), and their combinations CS+GC (purple), and the full $3\times2$pt, including BOSS and 2dFLenS galaxy-galaxy lensing (red).   For parameters constrained by the CMB, we also include constraints from \citet{planck/etal:2018} (green).}
		\label{fig:cosmology-params-all}
	\end{center}
\end{figure*}


Fig.~\ref{fig:S8comp_sensitivity} illustrates the results of a series of sensitivity tests, where we explore how our \tttp constraints on $S_8$ change when: 
we ignore the impact of baryon feedback (the `No baryon' case), fixing $A_{\rm baryon}=3.13$, corresponding to the non-linear matter power spectrum for a dark-matter only cosmology; 
we limit the analysis to a linear galaxy bias model, setting all higher-order bias terms in Eq. (\ref{eq:pgg}) to zero, as well as restricting the redshift-space distortion model to a Gaussian velocity distribution; 
and when we remove individual tomographic bins from our weak lensing observables. 
The only outlier in this series of tests is the linear-bias model, which highlights the importance of accurate non-linear galaxy bias modelling in \tttp analyses. 
This series of tests complements the more detailed KiDS-1000 internal consistency analysis of \citet{asgari/etal:inprep}, and is dissected in Appendix~\ref{app:sensitivity}.

\begin{figure}
	\begin{center}
		\includegraphics[width=\columnwidth]{Parameter_Plots/systematics/S8_comparison_blindC}
		\caption{\tttp constraints on $S_8$ for a series of sensitivity tests; when we ignore the impact of baryon feedback (the `No baryon' case), limit the analysis to a linear galaxy bias model (the `No higher order bias' case), and remove individual tomographic bins from our weak lensing observables.    
		\TT{Note that only the 3x2pt chain has the BOSS CosmoSIS-interface bug fixed, hence the overall offset - this will be fixed with the new runs!}
		\label{fig:S8comp_sensitivity}}
	\end{center}
\end{figure}

Table~\ref{tab:goodness-of-fit} records the goodness of fit for each component in our \tttp analysis.  The effective number of degrees of freedom (DoF) are calculated using the estimator described in section 6.3 of \citet{joachimi/etal:inprep}.  
The goodness of fit is excellent for the BOSS galaxy clustering.  For all other cases, the goodness of fit is acceptable\footnote{We define acceptable as $p \geq 0.001$, which corresponds to less than a $\sim 3\sigma$ event.   \citet{abbott/etal:2018} define acceptable as $\chi^2/{\rm DoF} < 1.4$.  We meet both these requirements.}, but only just.
We are unconcerned by these results, however, given the cosmic shear analysis of \citet{asgari/etal:inprep}, where a different choice in the cosmic shear two-point statistic results in an excellent goodness of fit, with no significant changes in the inferred cosmological parameters.    As such, we could be subject to an unlucky noise fluctuation that particularly impacts the band power estimator in Eq. (\ref{eq:cl_cosmicshear}).  Cautiously inspecting Fig.~\ref{fig:Pkk}, as `$\chi$-by-eye' is particularly dangerous with correlated data points, we nevertheless note a handful of outlying points, for example the low $\ell$-scales in the fifth tomographic bin.   We also note that \citet{giblin/etal:inprep} document a significant but low-level PSF residual systematic in the KiDS-1000 fourth and fifth tomographic bins that was shown to reduce the overall goodness of fit in a cosmic shear analysis, but not bias the recovered cosmological parameters \citep[see also the discussion in][]{amara/refregier:2008}.  Future work to remove these low-level residual distortions is therefore expected to further improve the goodness of fit.

\begin{table}
	\begin{center}
		\caption{The goodness of fit of the MAP cosmological model for each of the single and joint probe combinations with cosmic shear (CS), galaxy clustering (GC) and galaxy-galaxy lensing (GGL).   We list the $\chi^2$ value for each MAP model, the effective number of degrees of freedom (DoF) and the corresponding $p$-value which describes the probability of producing measurements that are more extreme than the data, assuming the MAP model is correct.   The DoF is presented as the difference between the total number of data points and the effective number of free parameters, accounting for the impact of priors and correlations between the parameters.}
		\label{tab:goodness-of-fit}
\begin{tabular}{lrcl}
    \toprule
    Probe             & $\chi^2$       & DoF       & $p$-value   \\
    \midrule
	CS               & $< 156.3$ & $120-4.5$ & 0.007 \\
	GC               & $< 169.5$ & $168-13$ & 0.202 \\
	CS+GGL           & $187.0$ & $142-7$ & 0.002 \\
	$3\times2$pt            & $367.8$ & $310-20$ & 0.001 \\

    \bottomrule
\end{tabular}
	\end{center}
\end{table}

\subsection{Comparison with Planck}
\label{sec:planck_comp}
\ch{@All - with apologies this section is the one most influenced by the outcome of the final chains.   The KiDS+BOSS+Planck chain has only just finishing running after 5 days on all our cores.   The iterated covariance \tttp KiDS-BOSS run will now take another 2 days to complete before this section can be finalised.  We decided not to wait to start consortium review for the rest of the paper - but you may wish to leave your review of this section until you've completed the review of the other parts, and hopefully by then we will also have completed this section.}

In our KiDS-1000-BOSS \tttp analysis, we find good agreement with Planck for the matter density parameter, $\Omega_{\rm m}$, and the Hubble parameter, $h$, (see Fig.~\ref{fig:cosmology-params}).
The amplitude of matter fluctuations, $\sigma_8$, that we infer from the clustering of galaxies within, and lensing by, the large-scale structure of the low-redshift Universe is lower than that inferred by Planck from the CMB, however. 

To quantify this discrepancy in the amplitude of matter fluctuations we concentrate on the parameter $S_{8} = \sigma_8 \sqrt{\Omega_{\rm m}/0.3}$ as it is tightly constrained and only exhibits negligible degeneracies, if at all, with the other cosmological parameters, $\Omega_{\rm m}$, $h$, and $n_{\rm s}$ \TT{the $n_{s}$ dependence still needs to be checked with separate \tttp chain}, as illustrated in Fig.~\ref{fig:cosmology-params-all}. 
%
The difference in the marginal posterior of $S_{8}$ between our \tttp analysis and the Planck \software{plik\_lite\_TTTEEE}+\software{lowl}+\software{lowE} likelihood, measured as the difference in the means divided by the standard deviations added in quadrature, is \kpoff.

\ch{Our results thus continue the trend of low-redshift probes preferring low amplitudes of matter fluctuations \citep{heymans/etal:2013, alam/etal:2017, abbott/etal:2018, hikage/etal:2019, wright/etal:2020b,DESclusters/etal:2020}. In these cases the reported low $S_8$, or $\sigma_8$, constraints are formally statistically consistent with Planck, and well below the detection of any anomalies at the $5\sigma$-level.   Considering, however, the $\sim 3\sigma$ difference that we have reported, and the overall trend amongst independent probes, we would argue that we have now reached an uncomfortable point when it comes to regarding the $S_8$ offset as a simple statistical fluke.}

\citet{Sanchez2020} pointed out that comparing $S_{8}$ between different experiments can be misleading due to the implicit dependence of $\sigma_{8}$ on $h$ through the $8\,h^{-1}{\rm Mpc}$ radius of the sphere within which the matter fluctuations are measured. 
A high value of $h$ would results in $\sigma_{8}$ measuring fluctuations within a smaller physical radius and thus inferring higher fluctuations. 
We thus also consider $S_{12} = \sigma_{12}\left(\omega_{\rm m}/0.14\right)^{0.4}$ \citep{Sanchez2020}, where $\sigma_{12}^{2}$ is the variance of the linear matter field at redshift zero in spheres of radius $12\,\mathrm{Mpc}$.
We find $S_{12} = 0.754^{+0.017}_{-0.015}$, with the value inferred by Planck being $S_{12} = 0.817_{-0.015}^{+0.011}
$, a discrepancy of $3.0\sigma$. 
This is in agreement with the results for $S_{8}$. 
Considering that our \tttp constraints on $h$ agree with those of Planck, and our $S_{8}$ constraints show no degeneracy with $h$, this is not surprising.

In light of the large parameter spaces that are being considered, focussing on a single parameter paints a simplistic picture on the agreement or disagreement between these probes, however. 
On a fundamental level, the question we wish to answer is whether a single model of the Universe can describe both the CMB as well as the low-redshift large-scale structure of the Universe.
Within our Bayesian inference framework, the Bayes factor provides a natural approach to model selection. 
The two models under consideration are 
\begin{description}
	\item[$\mathrm{M}_1$:] Both our \tttp data and Planck's measurements of the CMB are described by a single flat \LCDM cosmology.
	\item[$\mathrm{M}_2$:] The two data sets are described by different cosmologies for the low- and high redshift Universe, respectively.
\end{description}
The Bayes factor is then
\be
	R = \frac{P(\vec d | \mathrm{M}_1)P(\mathrm{M}_1)}{P(\vec d | \mathrm{M}_2)P(\mathrm{M}_2)} \ ,
\ee
where $P(\vec d | \mathrm{M}_1)$ is the probability of the data $\vec d$ under model $\mathrm{M}_1$ -- the Bayesian evidence. 
We assume the model priors $P(\mathrm{M}_1)$ and $P(\mathrm{M}_2)$ to be equal, that is, we make no a-priori assumption on the likelihood of $\mathrm{M}_1$ or $\mathrm{M}_2$. 
We find $R=X$, indicating XX according to the Jeffrey scale. 

\citet{Handley2019} discuss the dependence of $R$ on the parameter priors and propose a statistic $S$ -- called `suspiciousness' -- based on the Bayes factor but hardened against prior-dependences. \TT{expand by defining $S$ in more detail?} \ch{if we can compute d, then yes absolutely we need to explain this more.   If we can't then I propose we leave this as is, a niche area in the paper!}
We find $S=X$. 
For Gaussian posteriors, the quantity $d-2\log S$ is distributed as $\chi^2_{d}$, where $d$ is the difference in the model dimensionality between $\mathrm{M}_1$ and $\mathrm{M}_2$, thus allowing to assign a probability of the observed suspiciousness under the assumption that the two data sets are in concordance. 
\TT{Assuming we can compute $d$ accurately:} We calculate $d=XX$ and therefore conclude that our \tttp results and the Planck TTTEEE+lowl+lowE results are in discordance at the $XX\sigma$ level.
\TT{(Discuss numerical issues with getting $d$, mention other tension metrics, discuss restricting prior range to speed up sampling.) }


\subsection{Comparison with weak lensing surveys}
\label{sec:WL_comp}
Our results are consistent with weak lensing constraints in the literature.   We limit our discussion in this section to published \tttp analyses, referring the reader to \citet{asgari/etal:inprep} who discuss how the KiDS-1000 cosmic shear results compare with other weak lensing surveys.   We note that direct comparisons of cosmological parameters should be approached with some caution, as the priors adopted by different surveys and analyses are often informative \citep[see section 6.1 in][]{joachimi/etal:inprep}.   Homogenising priors for cosmic shear analyses, for example, has been shown to lead to different conclusions when assessing inter-survey consistency \citep{chang/etal:2019, joudaki/etal:2020, asgari/etal:2020_KD}.   

\begin{figure}
	\begin{center}
		\includegraphics[width=\columnwidth]{Parameter_Plots/cosmology/omegam_sigma8_h_blind_C}
		\caption{Marginalised posterior distributions for in $\sigma_8-\Omega_{\rm m}$ plane, comparing \tttp analysis from KiDS-1000 with DES Y1 \citep{abbott/etal:2018} and \citep{planck/etal:2018}.   The KiDS-1000-BOSS result can also be compared to our previous KV450-BOSS analysis from \citet{troester/etal:2020}. 
		\label{fig:DES_KiDS_comp}}
	\end{center}
\end{figure}


\citet{abbott/etal:2018} present the first year \tttp DES analysis (DES Y1), finding $S_8=0.773^{+0.026}_{-0.020}$, where they report the marginal posterior maximum and the tail credible intervals.  
This is in excellent agreement with our equivalent result, differing by $0.2\sigma$, with the KiDS-1000-BOSS error being 30\% tighter than the DES Y1 results.  The inclusion of BOSS to our \tttp analysis results in tight constraints on $\Omega_{\rm m}$.  
This leads to joint KiDS-1000-BOSS constraints on $\sigma_8=0.760^{+0.021}_{-0.023}$ that are more than twice as constraining compared to DES Y1 which finds $\sigma_8=0.817^{+0.045}_{-0.056}$, as shown in Fig.~\ref{fig:DES_KiDS_comp}. 
This comparison serves to highlight the additional power that can be extracted through the combination of spectroscopic and photometric surveys,  and the promising future for the planned overlap between the Dark Energy Spectroscopic Instrument survey \citep{DESI/etal:2016} and the 4-metre Multi-Object Spectroscopic Telescope \citep[4MOST,][]{guiglion/etal:2019},
with Euclid and the Vera C. Rubin Legacy Survey of Space and Time \citep{laureijs/etal:2011,lsst/etal:2009}, in addition to the nearer-term $\approx\!1400\,\mathrm{deg}^{2}$ of overlap between BOSS and the Hyper Suprime-Cam Strategic Program \citep[HSC,][]{aihara/etal:2019}. 

\citet{vanuitert/etal:2018} and \citet{joudaki/etal:2018} present \tttp analyses for the second KiDS release (KiDS-450), finding, respectively, $S_8 = 0.800_{-0.027}^{+0.029}$ (KiDS with GAMA) and $S_8 = 0.742 \pm 0.035$ (KiDS with BOSS and 2dFLenS limited to the overlap region). 
Both results are consistent with our KiDS-1000 results, noting that the increase in our $S_8$ constraining power, by a factor of $\approx\! 2$ in this analysis, is driven by increases in both the KiDS survey area, and the BOSS survey area.  

The impact of doubling the KiDS area can be seen by comparing to \citet{troester/etal:2020}, in Fig.~\ref{fig:DES_KiDS_comp}, who present a $2\times2$pt analyses for the KV450 KiDS release with the full BOSS area, finding $S_8 = 0.728 \pm 0.026$.   The $\approx\!40\%$ improvement in constraining power is consistent with expectations from the increased survey area, but a straightforward area-scaling comparison is inappropriate given that KiDS-1000 features improvements in the accuracy of the shear and photometric redshift calibrations, albeit at the expense of a decrease in the effective number density \citep[see][for details]{giblin/etal:inprep}.  
The offset of $1.2\sigma$ in $S_8$ between the KiDS-1000-BOSS and KV450-BOSS $S_8$ constraints reflects a number of differences in the analyses.  First, as the \tttp $S_8$ constraints are primarily driven by KiDS (see Fig.~\ref{fig:cosmology-params-all}), we expect a reasonable statistical fluctuation in this parameter given the increase in the KiDS survey area by a factor of two, even though the BOSS area remains fixed.   In contrast, BOSS primarily constrains $\Omega_{\rm m}$ which is impacted by the choice of prior on $n_{\rm s}$.  The wider $n_{\rm s}$ prior adopted in \citet{troester/etal:2020}, favours a slightly higher but less well-constrained value for $\Omega_{\rm m}$, leading to a slightly lower but less well-constrained value value for $\sigma_8$, when combined with cosmic shear.   If we had also chosen an uninformative prior on $n_{\rm s}$ for our KiDS-1000-BOSS analysis, a decision that we now cannot revise post unblinding, this would have likely served to further exacerbate any tension with the Planck CMB constraints.   
















\section{Conclusions}
\label{sec:conc}
In this analysis we have presented constraints on the flat $\Lambda$CDM cosmological model by combining observations of gravitational lensing and galaxy clustering to directly probe the evolution and distribution of the large-scale structures in the Universe.    Our survey of the $z \lesssim 1$ low-redshift Universe finds a matter distribution that is less clustered, compared to predictions from the best-fitting $\Lambda$CDM model to early-Universe CMB observations \citep{planck/etal:2018}.  This tendency for low-redshift probes to favour a smoother matter distribution compared to the CMB expectation has persisted since the first large-scale weak lensing survey \citep[CFHTLenS,][]{heymans/etal:2013}, but the significance of this effect has always been tantalisingly around, or below, the $\sim 3\sigma$ level.   It is therefore unclear if these differences are merely a statistical fluctuation, unaccounted for systematic errors, or a sign of interesting new physics.

Our new result does not lead to a resolution in the matter of statistical fluctuations, finding a \kpoff offset in the reported values for the structure growth parameter $S_8 = \sigma_8 \sqrt{\Omega_{\rm m}/0.3}$, where our best-fit value $S_8=$\kSeightval is \kpoffperc lower than the $S_8$ CMB constraint from \citet{planck/etal:2018}.   Through our series of image simulation analyses \citep{kannawadi/etal:2019}, catalogue null-tests \citep{giblin/etal:inprep}, variable depth mock galaxy survey analyses \citep{joachimi/etal:inprep}, optical-to-near-infrared photometric-spectroscopic redshift calibration, validated with mocks \citep{wright/etal:2020, vandenbusch/etal:2020,hildebrandt/etal:inprep}, internal consistency tests \citep{asgari/etal:inprep}, and marginalisation over a series of nuisance parameters that encompass our theoretical and calibration uncertainties,  we argue that we have, however, robustly assessed and accounted for all sources of systematic errors that are known about in the literature.    

In the next few years, two weak lensing surveys will see first light, with the launch of the {\it Euclid} satellite and the opening of the Vera~C.~Rubin Observatory.   These observatories will build the first two `full-sky' weak lensing surveys, which are highly complementary in terms of their differing strengths in depth and spatial resolution.  Combined with complementary overlapping redshift spectroscopy from DESI, 4MOST and Euclid, the multi-probe weak lensing and spectroscopic galaxy clustering methodology, which we have implemented in this analysis, provides a promising route forward for these next generation surveys.   We view this \tttp approach as just the start of the story, however, looking forward to a future combined analysis of weak lensing and galaxy clustering with both photometric and spectroscopic lenses, a combination which we call a `$6\times2$pt' approach \citep{bernstein:2009}.    This would allow for the optimal combination of information from the clustering cross-correlation of spectroscopic and photometric galaxies \citep{newman:2008}, an observable that we currently only use as an independent tool to validate our photometric redshift calibration \citep{hildebrandt/etal:inprep}.      Developments in the area of highly non-linear galaxy bias, baryon feedback and intrinsic alignment modelling, along with a sufficiently flexible but tractable redshift distribution model and an accurate `$6\times2$pt' covariance estimate, will all be required in order to realise this long-term goal.   The effort will, however, be worthwhile allowing for arguably the most optimal methodology to mitigate systematic errors, whilst simultaneously enhancing cosmological parameter constraints.

The ESO-KiDS public survey completed observations in July 2019, with the fifth and final data release, spanning $1350\,\mathrm{deg}^{2}$ of imaging anticipated before the end of 2020.  We therefore look forward to the completion of the KiDS weak lensing project, dubbed KiDS-Legacy, along with new results from the concurrent `Stage-III' surveys, DES and HSC, as the community prepares for the next exciting chapter with `full-sky' weak lensing surveys.  
\begin{appendix} 
\section{Galaxy properties for the KiDS-1000 sources and the BOSS and 2dFLenS lenses.}
\label{app:properties}

This Appendix tabulates the properties of the KiDS-1000 tomographic source samples, along with the properties of the BOSS and 2dFLenS lens samples, in Table~\ref{tab:datatab}.   We list the spectroscopic redshift, $z_{\rm min} < z_{\rm s} \leq z_{\rm max}$, selection for the lenses, and the photometric redshift, $z_{\rm min} < z_{\rm B} \leq z_{\rm max}$, selection for the sources, along with the mean redshift of each sample.  For the source sample, the true redshift distributions are estimated using the SOM methodology from \citet{wright/etal:2020}.     The shear calibration correction $m$, which can also be refereed to in the literature as the responsivity, $R = 1+m$, is listed for each source bin \citep{kannawadi/etal:2019}.  The effective number density of lenses and sources, per square arcmin,  defines the number of galaxies per square arcmin in the case of unit weights and, for the sources, unit responsivity \citep[see equations C.11 and C.13 in][]{joachimi/etal:inprep}.  We also list the  effective ellipticity dispersion $\sigma_{\epsilon,i}$, per ellipticity component, $i$, for each the weighted and calibrated source galaxy samples \citep[equation C.8 in][]{joachimi/etal:inprep}.

\begin{table}
\caption{Blind A: Galaxy properties for the BOSS and 2dFLenS lens (\lq L\rq) samples and the KiDS-1000 source (\lq S\rq) samples.}              % title of Table
\label{tab:datatab}      % is used to refer this table in the text
\centering                                      % used for centering table
\begin{tabular}{lcccccr}          % centered columns
\hline\hline                        % inserts double horizontal lines
ID & $z_{\rm min}$ &  $z_{\rm max}$& mean $z$ & $n_{\rm eff}$ & $\sigma_{\epsilon,i}$ & \multicolumn{1}{c}{$m$}\\    % table heading
\hline
\multicolumn{6}{l}{\bf KiDS-1000:}\\  
S1 & 0.1 & 0.3 & 0.26 & 0.57 &  0.27 & $-0.009\pm0.019$\\
S2 & 0.3 & 0.5 & 0.40 & 1.14 &  0.26 & $-0.011\pm0.020$\\
S3 & 0.5 & 0.7 & 0.56 & 1.82 &  0.28 & $-0.015\pm0.017$\\
S4 & 0.7 & 0.9 & 0.79 & 1.26 &  0.26 & $0.002\pm0.012$\\
S5 & 0.9 & 1.2 & 0.98 & 1.32 &  0.28 & $0.007\pm0.010$\\
\hline      
\multicolumn{6}{l}{\bf BOSS:}\\                             % inserts single horizontal line
L1 & 0.2 & 0.5 & 0.38 & $0.014$ & -  & \multicolumn{1}{c}{-}\\
L2 & 0.5 & 0.75 & 0.61 & $0.016$ & -  & \multicolumn{1}{c}{-}\\
\hline      
\multicolumn{6}{l}{\bf 2dFLenS:}\\                                % inserts single horizontal line
L1 & 0.2 & 0.5 & 0.36 & $0.006$ & - & \multicolumn{1}{c}{-}\\
L2 & 0.5 & 0.75 & 0.60 & $0.006$ & - & \multicolumn{1}{c}{-}\\
\hline
\end{tabular}
\end{table}

\begin{table}
\caption{Blind B: Galaxy properties for the BOSS and 2dFLenS lens (\lq L\rq) samples and the KiDS-1000 source (\lq S\rq) samples.}              % title of Table
\label{tab:datatab_BlindB}      % is used to refer this table in the text
\centering                                      % used for centering table
\begin{tabular}{lcccccr}          % centered columns
\hline\hline                        % inserts double horizontal lines
ID & $z_{\rm min}$ &  $z_{\rm max}$& mean $z$ & $n_{\rm eff}$ & $\sigma_{\epsilon,i}$ & \multicolumn{1}{c}{$m$}\\    % table heading
\hline      
\multicolumn{6}{l}{\bf KiDS-1000:}\\  
S1 & 0.1 & 0.3 & 0.26 & 0.57 &  0.27 & $-0.009\pm0.019$\\
S2 & 0.3 & 0.5 & 0.40 & 1.16 &  0.26 & $-0.011\pm0.020$\\
S3 & 0.5 & 0.7 & 0.56 & 1.87 &  0.27 & $-0.015\pm0.017$\\
S4 & 0.7 & 0.9 & 0.79 & 1.31 &  0.25 & $0.002\pm0.012$\\
S5 & 0.9 & 1.2 & 0.99 & 1.39 &  0.26 & $0.007\pm0.010$\\
\hline
\multicolumn{6}{l}{\bf BOSS:}\\                             % inserts single horizontal line
L1 & 0.2 & 0.5 & 0.38 & $0.014$ & -  & \multicolumn{1}{c}{-}\\
L2 & 0.5 & 0.75 & 0.61 & $0.016$ & -  & \multicolumn{1}{c}{-}\\
\hline      
\multicolumn{6}{l}{\bf 2dFLenS:}\\                                % inserts single horizontal line
L1 & 0.2 & 0.5 & 0.36 & $0.006$ & - & \multicolumn{1}{c}{-}\\
L2 & 0.5 & 0.75 & 0.60 & $0.006$ & - & \multicolumn{1}{c}{-}\\
\hline
\end{tabular}
\end{table}

\begin{table}
\caption{Blind C: Galaxy properties for the BOSS and 2dFLenS lens (\lq L\rq) samples and the KiDS-1000 source (\lq S\rq) samples.}              % title of Table
\label{tab:datatab_BlindC}      % is used to refer this table in the text
\centering                                      % used for centering table
\begin{tabular}{lcccccr}          % centered columns
\hline\hline                        % inserts double horizontal lines
ID & $z_{\rm min}$ &  $z_{\rm max}$& mean $z$ & $n_{\rm eff}$ & $\sigma_{\epsilon,i}$ & \multicolumn{1}{c}{$m$}\\    % table heading
\hline
\multicolumn{6}{l}{\bf KiDS-1000:}\\  
S1 & 0.1 & 0.3 & 0.26 & 0.57 &  0.27 & $-0.009\pm0.019$\\
S2 & 0.3 & 0.5 & 0.40 & 1.15 &  0.26 & $-0.011\pm0.020$\\
S3 & 0.5 & 0.7 & 0.56 & 1.84 &  0.27 & $-0.015\pm0.017$\\
S4 & 0.7 & 0.9 & 0.79 & 1.28 &  0.26 & $0.002\pm0.012$\\
S5 & 0.9 & 1.2 & 0.98 & 1.35 &  0.27 & $0.007\pm0.010$\\
\hline      
\multicolumn{6}{l}{\bf BOSS:}\\                             % inserts single horizontal line
L1 & 0.2 & 0.5 & 0.38 & $0.014$ & -  & \multicolumn{1}{c}{-}\\
L2 & 0.5 & 0.75 & 0.61 & $0.016$ & -  & \multicolumn{1}{c}{-}\\
\hline      
\multicolumn{6}{l}{\bf 2dFLenS:}\\                                % inserts single horizontal line
L1 & 0.2 & 0.5 & 0.36 & $0.006$ & - & \multicolumn{1}{c}{-}\\
L2 & 0.5 & 0.75 & 0.60 & $0.006$ & - & \multicolumn{1}{c}{-}\\
\hline
\end{tabular}
\end{table}




\section{Parameter priors}
\label{app:priors}
This Appendix tabulates the adopted KiDS-1000 priors and sampling parameters in Table~\ref{tab:priors}.   The uniform prior on the Hubble constant, $h$, reflects distance-ladder $\pm 5 \sigma$ constraints from \citet{riess/etal:2016}, which encompasses the value of $h$ favoured by \citet{planck/etal:2018}.  The uniform prior on the CDM density, $\omega_{\rm c}$, reflects Supernova Type Ia $\pm 5 \sigma$ constraints on $\Omega_{\rm m}$ from \citet{scolnic/etal:2018} combined with the extreme values of $h$ as allowed by our $h$-prior.   The uniform prior on the baryon density, $\omega_{\rm b}$, reflects big bang nucleosynthesis $\pm 5 \sigma$ constraints from \citet{olive/etal:2014}.   As discussed in Section~\ref{sec:KCAP} we choose to sample with an uninformative uniform prior on $S_8$ to avoid implicit informative priors from a uniform prior on the primordial power spectrum amplitude $A_{\rm s}$.    The uniform prior on the scalar spectral index, $n_{\rm s}$, reflects a restriction in our likelihood implementation, where the \citet{sanchez/etal:2017} galaxy clustering likelihood becomes prohibitively slow for $n_{\rm s}>1.1$.  With the upper limit of the top-hat prior fixed by this computational limitation, we choose to symmetrise the prior around the theoretical expectation of $n_{\rm s}=0.97$.  

Turning to astrophysical priors, the galaxy bias parameter top-hat priors, $b_1$, $b_2$,  $\gamma_3^-$, and $a_{\rm vir}$ match those adopted in \citet{sanchez/etal:2017}, with independent sets of parameters for each of the two BOSS redshift slices.   Wide uniform priors for the intrinsic alignment parameter $A_{\rm IA}$ are chosen to be uninformative.    Uniform priors on the baryon feedback parameter $A_{\rm bary}$ are chosen such that the resulting \citet{mead/etal:2015} model of the non-linear matter power spectrum encompasses both the most aggressive feedback model from the \citet{vandaalen/etal:2011} suite of hydrodynamical simulations, along with the dark matter-only case.

There are five additional correlated nuisance parameters, $\delta^i_z$, that model uncertainty in the mean of the source redshift distributions.  We adopt a multivariate Gaussian prior for the vector $\vek{\delta}_z$ with a mean $\vek{\mu} = (0.0001,0.0021,0.0129,0.0110,-0.0060)$, and a co-variance, $C_{\delta z}$, as calibrated using mock galaxy catalogues in \citet{wright/etal:2020}.   The diagonal terms of $\vek{C}_{\delta z}$ are typically at the level of $\sim(0.01)^2$, with correlations between the different redshift bins, i.e the ratio of the off-diagonal to diagonal terms,  ranging from zero to $\sim 0.3$ \citep[see section 3.3 of][for details]{joachimi/etal:inprep}.


\begin{table}
\caption{KiDS-1000 sampling parameters and priors.}              % title of Table
\label{tab:priors}      % is used to refer this table in the text
\centering                                      % used for centering table
\begin{tabular}{lll}          % centered columns (4 columns)
\hline\hline                        % inserts double horizontal lines
Parameter & Symbol & Prior \\    % table heading
\hline                                   % inserts single horizontal line
Hubble constant & $h$ & $\bb{0.64,\,0.82}$ \\
CDM density & $\omega_{\rm c}$ & $\bb{0.051,\,0.255}$ \\
Baryon density & $\omega_{\rm b}$ & $\bb{0.019,\,0.026}$ \\
Density fluctuation amp. & $S_8$ & $\bb{0.1,\,1.3}$ \\
Scalar spectral index & $n_{\rm s}$ & $\bb{0.84,\,1.1}$ \\
\hline
Linear galaxy bias & $b_1 \;[2]$ & $\bb{0.5,\,9}$ \\
Quadratic galaxy bias & $b_2 \;[2]$ & $\bb{-0.6,\,4}$ \\
Non-local galaxy bias & ${\gamma_3^-} \;[2]$ & $\bb{-3,\,3}$ \\
Virial velocity parameter & $a_{\rm vir} \;[2]$ & $\bb{0,\,12}$ \\
Intrinsic alignment amp. & $A_{\rm IA}$ & $\bb{-6,\,6}$ \\
Baryon feedback amp. & $A_{\rm bary}$ & $\bb{2,\,3.13}$ \\
\hline
Redshift offsets & ${\bf \delta_z}$ & ${\cal N}(\vek{\mu};\vek{C}_{\delta z})$ \\
\hline
\end{tabular}
\tablefoot{Primary cosmological parameters for the flat $\Lambda$CDM model are listed in the first section. The second section lists astrophysical nuisance parameters to model galaxy bias (with independent parameters for each of the two BOSS redshift bins as indicated with the bracket $[2]$), intrinsic galaxy alignments, and baryon feedback.  Observational redshift nuisance parameters are listed in the final section. Prior values in square brackets are the limits of the adopted uniform top-hat priors.  ${\cal N}(\mu;C)$ corresponds to a five dimensional multivariate Gaussian prior with mean $\vek{\mu}$ and covariance $\vek{C}_{\delta z}$.}
\end{table}

\end{appendix}

\begin{acknowledgements}
We thank Mike Jarvis for his continuing enhancements, clear documentation and maintenance of the excellent {\sc TreeCorr} software package and our external blinder Matthias Bartelmann who revealed the key for which of the three catalogues analysed was the true unblinded catalogue at the end of the KiDS-1000 study.   We also wish to thank the Vera C. Rubin Observatory LSST-DESC Software Review Policy Committee (Camille Avestruz, Matt Becker, Celine Combet, Mike Jarvis, David Kirkby, Joe Zuntz with CH) for their Software Policy document which we followed, to the best of our abilities, during the KiDS-1000 project.   Following this policy the software used to carry out the various analyses presented in this paper will be made public on publication of this paper.\\

This project has received funding from the European Union's Horizon 2020 research and innovation programme: We acknowledge support from the European Research Council under grant agreement No.~647112 (CH, TT, MA, CL and BG). TT also acknowledges support under the Marie Sk\l{}odowska-Curie grant agreement No.~797794. CH acknowledges support from the Max Planck Society and the Alexander von Humboldt Foundation in the framework of the Max Planck-Humboldt Research Award endowed by the Federal Ministry of Education and Research. HH is supported by a Heisenberg grant of the Deutsche Forschungsgemeinschaft (Hi 1495/5-1). AK acknowledges support from Vici grant 639.043.512, financed by the Netherlands Organisation for Scientific Research (NWO). KK acknowledges support by the Alexander von Humboldt Foundation.\\
%
The results in this paper are based on observations made with ESO Telescopes at the La Silla Paranal Observatory under programme IDs 177.A-3016, 177.A-3017, 177.A-3018 and 179.A-2004, and on data products produced by the KiDS consortium. The KiDS production team acknowledges support from: Deutsche Forschungsgemeinschaft, ERC, NOVA and NWO-M grants; Target; the University of Padova, and the University Federico II (Naples).\bg{@KK - do we need to add in any VISTA acks?.}\\

The BOSS-related results in this paper have been made possible thanks to SDSS-III. Funding for SDSS-III has been provided by the Alfred P. Sloan Foundation, the Participating Institutions, the National Science Foundation, and the U.S. Department of Energy Office of Science.   SDSS-III is managed by the Astrophysical Research Consortium for the Participating Institutions of the SDSS-III Collaboration including the University of Arizona, the Brazilian Participation Group, Brookhaven National Laboratory, Carnegie Mellon University, University of Florida, the French Participation Group, the German Participation Group, Harvard University, the Instituto de Astrofisica de Canarias, the Michigan State/Notre Dame/JINA Participation Group, Johns Hopkins University, Lawrence Berkeley National Laboratory, Max Planck Institute for Astrophysics, Max Planck Institute for Extraterrestrial Physics, New Mexico State University, New York University, Ohio State University, Pennsylvania State University, University of Portsmouth, Princeton University, the Spanish Participation Group, University of Tokyo, University of Utah, Vanderbilt University, University of Virginia, University of Washington, and Yale University.\\

The 2dFLenS-related results are based on data acquired through the Australian Astronomical Observatory, under program A/2014B/008. It would not have been possible without the dedicated work of the staff of the AAO in the development and support of the 2dF-AAOmega system, and the running of the AAT.\\

{ {\it Author contributions:}  All authors contributed to the development and writing of this paper.  The authorship list is given in three groups:  the lead authors (CH \& TT) followed by two alphabetical groups.  The first alphabetical group includes those who are key contributors to both the scientific analysis and the data products.  The second group covers those who have either made a significant contribution to the data products, or to the scientific analysis.}
\end{acknowledgements}


\bibliographystyle{aa} % style aa.bst
\bibliography{references} % your references 


%-------------------------------------------------------------------


\end{document}

