\begin{appendix} 
\section{Galaxy properties for the KiDS-1000 sources and the BOSS and 2dFLenS lenses.}
\label{app:properties}

This Appendix tabulates the properties of the KiDS-1000 tomographic source samples, along with the properties of the BOSS and 2dFLenS lens samples, in Table~\ref{tab:datatab}.   We list the spectroscopic redshift, $z_{\rm min} < z_{\rm s} \leq z_{\rm max}$, selection for the lenses, and the photometric redshift, $z_{\rm min} < z_{\rm B} \leq z_{\rm max}$, selection for the sources, along with the mean redshift of each sample.  For the source sample, the true redshift distributions are estimated using the SOM methodology from \citet{wright/etal:2020}.     The shear calibration correction $m$, which can also be refereed to in the literature as the responsivity, $R = 1+m$, is listed for each source bin \citep{kannawadi/etal:2019}.  The effective number density of lenses and sources, per square arcmin,  defines the number of galaxies per square arcmin in the case of unit weights and, for the sources, unit responsivity \citep[see equations C.11 and C.13 in][]{joachimi/etal:inprep}.  We also list the  effective ellipticity dispersion $\sigma_{\epsilon,i}$, per ellipticity component, $i$, for each the weighted and calibrated source galaxy samples \citep[equation C.8 in][]{joachimi/etal:inprep}.

\begin{table}
\caption{Blind A: Galaxy properties for the BOSS and 2dFLenS lens (\lq L\rq) samples and the KiDS-1000 source (\lq S\rq) samples.}              % title of Table
\label{tab:datatab}      % is used to refer this table in the text
\centering                                      % used for centering table
\begin{tabular}{lcccccr}          % centered columns
\hline\hline                        % inserts double horizontal lines
ID & $z_{\rm min}$ &  $z_{\rm max}$& mean $z$ & $n_{\rm eff}$ & $\sigma_{\epsilon,i}$ & \multicolumn{1}{c}{$m$}\\    % table heading
\hline
\multicolumn{6}{l}{\bf KiDS-1000:}\\  
S1 & 0.1 & 0.3 & 0.26 & 0.57 &  0.27 & $-0.009\pm0.019$\\
S2 & 0.3 & 0.5 & 0.40 & 1.14 &  0.26 & $-0.011\pm0.020$\\
S3 & 0.5 & 0.7 & 0.56 & 1.82 &  0.28 & $-0.015\pm0.017$\\
S4 & 0.7 & 0.9 & 0.79 & 1.26 &  0.26 & $0.002\pm0.012$\\
S5 & 0.9 & 1.2 & 0.98 & 1.32 &  0.28 & $0.007\pm0.010$\\
\hline      
\multicolumn{6}{l}{\bf BOSS:}\\                             % inserts single horizontal line
L1 & 0.2 & 0.5 & 0.38 & $0.014$ & -  & \multicolumn{1}{c}{-}\\
L2 & 0.5 & 0.75 & 0.61 & $0.016$ & -  & \multicolumn{1}{c}{-}\\
\hline      
\multicolumn{6}{l}{\bf 2dFLenS:}\\                                % inserts single horizontal line
L1 & 0.2 & 0.5 & 0.36 & $0.006$ & - & \multicolumn{1}{c}{-}\\
L2 & 0.5 & 0.75 & 0.60 & $0.006$ & - & \multicolumn{1}{c}{-}\\
\hline
\end{tabular}
\end{table}

\begin{table}
\caption{Blind B: Galaxy properties for the BOSS and 2dFLenS lens (\lq L\rq) samples and the KiDS-1000 source (\lq S\rq) samples.}              % title of Table
\label{tab:datatab_BlindB}      % is used to refer this table in the text
\centering                                      % used for centering table
\begin{tabular}{lcccccr}          % centered columns
\hline\hline                        % inserts double horizontal lines
ID & $z_{\rm min}$ &  $z_{\rm max}$& mean $z$ & $n_{\rm eff}$ & $\sigma_{\epsilon,i}$ & \multicolumn{1}{c}{$m$}\\    % table heading
\hline      
\multicolumn{6}{l}{\bf KiDS-1000:}\\  
S1 & 0.1 & 0.3 & 0.26 & 0.57 &  0.27 & $-0.009\pm0.019$\\
S2 & 0.3 & 0.5 & 0.40 & 1.16 &  0.26 & $-0.011\pm0.020$\\
S3 & 0.5 & 0.7 & 0.56 & 1.87 &  0.27 & $-0.015\pm0.017$\\
S4 & 0.7 & 0.9 & 0.79 & 1.31 &  0.25 & $0.002\pm0.012$\\
S5 & 0.9 & 1.2 & 0.99 & 1.39 &  0.26 & $0.007\pm0.010$\\
\hline
\multicolumn{6}{l}{\bf BOSS:}\\                             % inserts single horizontal line
L1 & 0.2 & 0.5 & 0.38 & $0.014$ & -  & \multicolumn{1}{c}{-}\\
L2 & 0.5 & 0.75 & 0.61 & $0.016$ & -  & \multicolumn{1}{c}{-}\\
\hline      
\multicolumn{6}{l}{\bf 2dFLenS:}\\                                % inserts single horizontal line
L1 & 0.2 & 0.5 & 0.36 & $0.006$ & - & \multicolumn{1}{c}{-}\\
L2 & 0.5 & 0.75 & 0.60 & $0.006$ & - & \multicolumn{1}{c}{-}\\
\hline
\end{tabular}
\end{table}

\begin{table}
\caption{Blind C: Galaxy properties for the BOSS and 2dFLenS lens (\lq L\rq) samples and the KiDS-1000 source (\lq S\rq) samples.}              % title of Table
\label{tab:datatab_BlindC}      % is used to refer this table in the text
\centering                                      % used for centering table
\begin{tabular}{lcccccr}          % centered columns
\hline\hline                        % inserts double horizontal lines
ID & $z_{\rm min}$ &  $z_{\rm max}$& mean $z$ & $n_{\rm eff}$ & $\sigma_{\epsilon,i}$ & \multicolumn{1}{c}{$m$}\\    % table heading
\hline
\multicolumn{6}{l}{\bf KiDS-1000:}\\  
S1 & 0.1 & 0.3 & 0.26 & 0.57 &  0.27 & $-0.009\pm0.019$\\
S2 & 0.3 & 0.5 & 0.40 & 1.15 &  0.26 & $-0.011\pm0.020$\\
S3 & 0.5 & 0.7 & 0.56 & 1.84 &  0.27 & $-0.015\pm0.017$\\
S4 & 0.7 & 0.9 & 0.79 & 1.28 &  0.26 & $0.002\pm0.012$\\
S5 & 0.9 & 1.2 & 0.98 & 1.35 &  0.27 & $0.007\pm0.010$\\
\hline      
\multicolumn{6}{l}{\bf BOSS:}\\                             % inserts single horizontal line
L1 & 0.2 & 0.5 & 0.38 & $0.014$ & -  & \multicolumn{1}{c}{-}\\
L2 & 0.5 & 0.75 & 0.61 & $0.016$ & -  & \multicolumn{1}{c}{-}\\
\hline      
\multicolumn{6}{l}{\bf 2dFLenS:}\\                                % inserts single horizontal line
L1 & 0.2 & 0.5 & 0.36 & $0.006$ & - & \multicolumn{1}{c}{-}\\
L2 & 0.5 & 0.75 & 0.60 & $0.006$ & - & \multicolumn{1}{c}{-}\\
\hline
\end{tabular}
\end{table}




\section{Parameter priors}
\label{app:priors}
This Appendix tabulates the adopted KiDS-1000 priors and sampling parameters in Table~\ref{tab:priors}.   The uniform prior on the Hubble constant, $h$, reflects distance-ladder $\pm 5 \sigma$ constraints from \citet{riess/etal:2016}, which encompasses the value of $h$ favoured by \citet{planck/etal:2018}.  The uniform prior on the CDM density, $\omega_{\rm c}$, reflects Supernova Type Ia $\pm 5 \sigma$ constraints on $\Omega_{\rm m}$ from \citet{scolnic/etal:2018} combined with the extreme values of $h$ as allowed by our $h$-prior.   The uniform prior on the baryon density, $\omega_{\rm b}$, reflects big bang nucleosynthesis $\pm 5 \sigma$ constraints from \citet{olive/etal:2014}.   As discussed in Section~\ref{sec:KCAP} we choose to sample with an uninformative uniform prior on $S_8$ to avoid implicit informative priors from a uniform prior on the primordial power spectrum amplitude $A_{\rm s}$.    The uniform prior on the scalar spectral index, $n_{\rm s}$, reflects a restriction in our likelihood implementation, where the \citet{sanchez/etal:2017} galaxy clustering likelihood becomes prohibitively slow for $n_{\rm s}>1.1$.  With the upper limit of the top-hat prior fixed by this computational limitation, we choose to symmetrise the prior around the theoretical expectation of $n_{\rm s}=0.97$.  

Turning to astrophysical priors, the galaxy bias parameter top-hat priors, $b_1$, $b_2$,  $\gamma_3^-$, and $a_{\rm vir}$ match those adopted in \citet{sanchez/etal:2017}, with independent sets of parameters for each of the two BOSS redshift slices.   Wide uniform priors for the intrinsic alignment parameter $A_{\rm IA}$ are chosen to be uninformative.    Uniform priors on the baryon feedback parameter $A_{\rm bary}$ are chosen such that the resulting \citet{mead/etal:2015} model of the non-linear matter power spectrum encompasses both the most aggressive feedback model from the \citet{vandaalen/etal:2011} suite of hydrodynamical simulations, along with the dark matter-only case.

There are five additional correlated nuisance parameters, $\delta^i_z$, that model uncertainty in the mean of the source redshift distributions.  We adopt a multivariate Gaussian prior for the vector $\vek{\delta}_z$ with a mean $\vek{\mu} = (0.0001,0.0021,0.0129,0.0110,-0.0060)$, and a co-variance, $C_{\delta z}$, as calibrated using mock galaxy catalogues in \citet{wright/etal:2020}.   The diagonal terms of $\vek{C}_{\delta z}$ are typically at the level of $\sim(0.01)^2$, with correlations between the different redshift bins, i.e the ratio of the off-diagonal to diagonal terms,  ranging from zero to $\sim 0.3$ \citep[see section 3.3 of][for details]{joachimi/etal:inprep}.


\begin{table}
\caption{KiDS-1000 sampling parameters and priors.}              % title of Table
\label{tab:priors}      % is used to refer this table in the text
\centering                                      % used for centering table
\begin{tabular}{lll}          % centered columns (4 columns)
\hline\hline                        % inserts double horizontal lines
Parameter & Symbol & Prior \\    % table heading
\hline                                   % inserts single horizontal line
Hubble constant & $h$ & $\bb{0.64,\,0.82}$ \\
CDM density & $\omega_{\rm c}$ & $\bb{0.051,\,0.255}$ \\
Baryon density & $\omega_{\rm b}$ & $\bb{0.019,\,0.026}$ \\
Density fluctuation amp. & $S_8$ & $\bb{0.1,\,1.3}$ \\
Scalar spectral index & $n_{\rm s}$ & $\bb{0.84,\,1.1}$ \\
\hline
Linear galaxy bias & $b_1 \;[2]$ & $\bb{0.5,\,9}$ \\
Quadratic galaxy bias & $b_2 \;[2]$ & $\bb{-0.6,\,4}$ \\
Non-local galaxy bias & ${\gamma_3^-} \;[2]$ & $\bb{-3,\,3}$ \\
Virial velocity parameter & $a_{\rm vir} \;[2]$ & $\bb{0,\,12}$ \\
Intrinsic alignment amp. & $A_{\rm IA}$ & $\bb{-6,\,6}$ \\
Baryon feedback amp. & $A_{\rm bary}$ & $\bb{2,\,3.13}$ \\
\hline
Redshift offsets & ${\bf \delta_z}$ & ${\cal N}(\vek{\mu};\vek{C}_{\delta z})$ \\
\hline
\end{tabular}
\tablefoot{Primary cosmological parameters for the flat $\Lambda$CDM model are listed in the first section. The second section lists astrophysical nuisance parameters to model galaxy bias (with independent parameters for each of the two BOSS redshift bins as indicated with the bracket $[2]$), intrinsic galaxy alignments, and baryon feedback.  Observational redshift nuisance parameters are listed in the final section. Prior values in square brackets are the limits of the adopted uniform top-hat priors.  ${\cal N}(\mu;C)$ corresponds to a five dimensional multivariate Gaussian prior with mean $\vek{\mu}$ and covariance $\vek{C}_{\delta z}$.}
\end{table}

\end{appendix}