\section{Conclusions}
\label{sec:conc}
In this analysis we have presented constraints on the flat $\Lambda$CDM cosmological model by combining observations of gravitational lensing and galaxy clustering to directly probe the evolution and distribution of the large-scale structures in the Universe.    Our survey of the $z \lesssim 1$ low-redshift Universe finds a matter distribution that is less clustered, compared to predictions from the best-fitting $\Lambda$CDM model to early-Universe CMB observations \citep{planck/etal:2018}.  This tendency for low-redshift probes to favour a smoother matter distribution compared to the CMB expectation has persisted since the first large-scale weak lensing survey \citep[CFHTLenS,][]{heymans/etal:2013}, but the significance of this effect has always been tantalisingly around, or below, the $\sim\! 3\,\sigma$ level.   It is therefore unclear if these differences are merely a statistical fluctuation, unaccounted for systematic errors, or a sign of interesting new physics.

Our new result does not lead to a resolution in the matter of statistical fluctuations, finding a \kpoff offset in the structure growth parameter $S_8 = \sigma_8 \sqrt{\Omega_{\rm m}/0.3}$ with $S_8=$\kSeightval.  Comparing the marginal $S_8$ constraints, we find $S_8$ to be \kpoffperc lower than the CMB constraint from \citet{planck/etal:2018}.   For a series of `tension' metrics that quantify differences in terms of the full posterior distributions, we find that the KiDS-1000 and {\it Planck} results agree at the $\sim\! 2\,\sigma$ level.   Through our series of image simulation analyses \citep{kannawadi/etal:2019}, catalogue null-tests \citep{giblin/etal:inprep}, variable depth mock galaxy survey analyses \citep{joachimi/etal:inprep}, optical-to-near-infrared photometric-spectroscopic redshift calibration, validated with mocks \citep{wright/etal:2020, vandenbusch/etal:2020,hildebrandt/etal:inprep}, internal consistency tests \citep[][Fig.~\ref{fig:cosmology-params-all} and Appendix~\ref{app:sensitivity}]{asgari/etal:inprep}, and marginalisation over a series of nuisance parameters that encompass our theoretical and calibration uncertainties (Appendix~\ref{app:priors}),  we argue that we have, however, addressed the question of \tttp systematic errors, robustly assessing and accounting for all sources of systematics that are known about in the literature.    

The KiDS-1000 cosmic shear constraints are highly complementary to the BOSS galaxy clustering constraints, leading to tight constraints in our joint \tttp analysis that are more than twice as constraining for the matter fluctuation amplitude parameter, $\sigma_8 = 0.760^{+0.021}_{-0.023}$, compared to previous \tttp analyses.    In the future, analysis of the clustering and galaxy-galaxy lensing of photometric samples with very accurate photometric redshifts \citep[see for example][]{vakili/etal:2019}, presents an opportunity for a future alternative KiDS-only \tttp photometric analysis, similar to the approach taken in \citet{abbott/etal:2018}.

In the next few years, two weak lensing surveys will see first light, with the launch of the {\it Euclid} satellite and the opening of the Vera~C.~Rubin Observatory.   These observatories will build the first two `full-sky' weak lensing surveys, which are highly complementary in terms of their differing strengths in depth and spatial resolution\footnote{The space-based {\it Nancy Grace Roman} Telescope is currently scheduled for launch in 2025 \citep{akeson/etal:2019}, joining {\it Euclid} and {\it Rubin} as an optimal weak lensing observatory for the future.}.  Combined with complementary overlapping redshift spectroscopy from DESI, 4MOST and {\it Euclid}, the multi-probe weak lensing and spectroscopic galaxy clustering methodology, which we have implemented in this analysis, provides a promising route forward for these next generation surveys.   We view this \tttp approach as just the start of the story, however, looking forward to a future combined analysis of weak lensing and galaxy clustering with both photometric and spectroscopic lenses, a combination which we call a `$6\times2$pt' approach \citep{bernstein:2009}.    This would allow for the optimal combination of information from the clustering cross-correlation of spectroscopic and photometric galaxies \citep{newman:2008}, an observable that we currently only use as an independent tool to validate our photometric redshift calibration \citep{hildebrandt/etal:inprep}.      Developments in the area of highly non-linear galaxy bias, baryon feedback and intrinsic alignment modelling, along with a sufficiently flexible but tractable redshift distribution model and an accurate `$6\times2$pt' covariance estimate, will all be required in order to realise this long-term goal.   The effort will, however, be worthwhile allowing for the implementation of arguably the most robust methodology available to mitigate systematic errors, whilst simultaneously enhancing cosmological parameter constraints.

The ESO-KiDS public survey completed observations in July 2019, spanning $1350\,\mathrm{deg}^{2}$.   We therefore look forward to the fifth and final KiDS data release, `KiDS-Legacy', along with new results from the concurrent `Stage-III' surveys, DES and HSC, whilst the community prepares for the next exciting chapter of `full-sky' weak lensing surveys.  