\section{Conclusions and summary}
\label{sec:conc}
In this analysis we have presented constraints on the flat $\Lambda$CDM cosmological model by combining observations of gravitational lensing and galaxy clustering to directly probe the evolution and distribution of the large-scale structures in the Universe.    Our survey of the $z \lesssim 1$ low-redshift Universe finds a matter distribution that is less clustered compared to predictions for the present-day Universe, taken from the best-fitting $\Lambda$CDM model to early-Universe CMB observations from \citet{planck/etal:2018}.  This tendency for low-redshift probes to favour a smoother matter distribution compared to the CMB expectation, has persisted since the first large-scale weak lensing surveys \citep{heymans/etal:2013}, but the significance of this effect has always been tantalisingly below the $\sim 3\sigma$ level.   It is therefore unclear if these differences are merely a statistical fluctuation, or a sign of interesting new physics.    Our new result unfortunately does not resolve this matter, finding an $X\sigma$ offset in the reported values for the structure growth parameter $S_8$.   Through our series of image simulation analyses \citep{kannawadi/etal:2019}, catalogue null-tests \citep{giblin/etal:inprep}, variable depth mock galaxy survey analyses \citep{joachimi/etal:inprep}, optical-to-near-infrared photometric-spectroscopic redshift calibration, validated with mocks \citep{wright/etal:2019, vandenbusch/etal:inprep,hildebrandt/etal:inprep}, and internal consistency tests \citep{asgari/etal:inprep}, we argue that we have however robustly addressed the question of systematics, confirming that these differences could not arise from significant systematic errors in our analysis.


