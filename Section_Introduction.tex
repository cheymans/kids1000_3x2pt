\section{Introduction}
\label{sec:intro}

Observations of the cosmic microwave background (CMB) have delivered high-precision
constraints for the cosmological parameters of the flat, cold dark
matter and cosmological constant model of the Universe
\citep[$\Lambda$CDM,][]{planck/etal:2018}.  With only six free
parameters, this flat $\Lambda$CDM model provides an exquisite fit to observations of
the anisotropies in the CMB.    The same model predicts a range of
different observables in the present day Universe, including the cosmic expansion rate \citep{weinberg/1972}, and the
distribution of, and gravitational lensing by, large-scale
structures \citep{peebles/1980,bartelmann/schneider:2001,eisenstein/etal:2005}.  
In most cases there
is agreement between the measured cosmological parameters of the flat $\Lambda$CDM model, when comparing those
constrained at the CMB epoch with those constrained through a variety of
lower-redshift probes \citep[see the discussion in][and references
therein]{planck/etal:2018}.   Recent increases in the statistical
precision of the lower-redshift probes has,
however, revealed some statistically significant differences.  Most
notably a $4.4\sigma$ difference in the value of the Hubble constant,
$H_0$, has been reported in \citet{riess/etal:2019}.  If this difference cannot be
attributed to systematic errors in either, or both, experiment, this result
suggests that the flat $\Lambda$CDM model is incomplete.

Many
extensions have been proposed to reconcile the observed differences between
high- and low-redshift probes \citep[see for
example][]{poulin/etal:2018,divalentino/etal:2020}.  Each, however, require
additional components
to the cosmological model that move it even further away from the
standard model of particle physics, a model that already struggles to motivate
the existence of cold dark matter and a cosmological constant.  
As the statistical power of the observations continues to improve, focus has moved to establishing
full understanding of all systematic errors, and the development of mitigation approaches, 
in preparation for the high-precision `full-sky' imaging and spectroscopic
cosmology surveys of the 2020's \citep[{\it Euclid},][]{laureijs/etal:2011,lsst/etal:2009,DESI/etal:2016}.

In this analysis we present a multi-probe `same-sky' analysis of the
evolution of large-scale structures, using imaging and spectroscopic surveys.
Our first observable is the weak gravitational lensing of background
galaxies by foreground large-scale
structures, 
known as `{\it cosmic shear}'.    Our second
observable is the {\it anisotropic clustering of galaxies} within these
large-scale structures, combining measurements of both redshift-space
distortions and baryon acoustic
oscillations.   Our third observable is the weak gravitational lensing of background
galaxies by foreground galaxies, known as
`{\it galaxy-galaxy lensing}'.   As these three sets of two-point
statistics are analysed simultaneously, this combination of probes
is usually referred to as a `\tttp' analysis. 

Each observable in our multi-probe analysis is subject to systematic
uncertainties.  For a cosmic shear analysis, the observable is a
combination of the true cosmological signal with a low-level signal
arising from the intrinsic alignment of galaxies, as well as potential residual
correlations in the data induced by the atmosphere, telescope and
camera.   The signal can also be scaled by both shear
 and photometric redshift measurement calibration errors
 \citep[see][and references therein]{mandelbaum:2018}.   For a galaxy
   clustering analysis, the observable is the true
   cosmological signal modulated by an uncertain, non-linear and
   evolving, galaxy bias function.  This function maps how
   the galaxies trace the
   underlying total matter distribution \citep[see][and references
   therein]{desjacques/etal:2018}. 
   The cosmological clustering
   also needs to be accurately distinguished from artificial clustering in the galaxy sample,
   arising from potentially uncharacterised inhomogeneities in the target selection \citep[see for example][]{ross/etal:2012}. 
   Finally, the galaxy-galaxy
   lensing analysis is subject to the systematics that impact both the
   cosmic shear and clustering analyses.

   When analysing these
   observables in combination
   the different astrophysical and systematic dependencies allow for some degree of
   self-calibration \citep{bernstein/jain:2004, hu/jain:2004,
     bernstein:2009,joachimi/bridle:2010}.  `Same-sky'
   surveys, in which imaging for weak lensing observables overlaps with
   spectroscopy for anisotropic galaxy clustering measurements,
   also allows for their cross-correlation.  Such a survey design therefore
   presents a robust
   cosmological tool that can calibrate and mitigate systematic and astrophysical
   uncertainties through a series of nuisance parameters.   In
   addition to enhanced control over systematics, this combination of probes
   breaks cosmological parameter degeneracies from each individual
   probe. 
   For a flat $\Lambda$CDM model
   this leads to significantly tighter constraints on the matter fluctuation amplitude 
   parameter, $\sigma_8$, and the matter density parameter, $\Omega_{\rm m}$, whilst also decreasing the
   uncertainty on the recovered dark energy equation of state
   parameter in extended cosmology scenarios \citep{hu/jain:2004,gaztanaga/etal:2012}.
   
   Three variants of a joint `\tttp' analysis have been
   conducted to date.  \citet{vanuitert/etal:2018} present a joint power-spectrum
   analysis of the Kilo-Degree Survey \citep[KiDS,][]{kuijken/etal:2015} with
   the Galaxies And Mass Assembly survey
   \citep[GAMA,][]{liske/etal:2015}, incorporating projected
   angular clustering measurements.   \citet{joudaki/etal:2018}
   present a joint analysis of KiDS with the
   2-degree Field Lensing Survey \citep[2dFLenS,][]{blake/etal:2016}
   and the overlapping area in the Baryon Oscillation Spectroscopic Survey \citep[BOSS,][]{alam/etal:2015}, incorporating
   redshift-space clustering measurements.  \citet{abbott/etal:2018}
   present a joint real-space lensing-clustering analysis of the Dark
   Energy Survey \citep[DES,][]{drlicawagner/etal:2018}, using a high-quality
   photometric redshift sample of luminous red galaxies for their projected
   angular clustering measurements.  
   In all three cases a
   linear galaxy bias model was adopted. 
   
   In this analysis we enhance and build upon the advances of previous `\tttp' studies.   We analyse the most recent KiDS data release \citep[KiDS-1000,][]{kuijken/etal:2019}, more than doubling the
   survey area from previous KiDS analyses.   We utilise the full BOSS
   area and the `full-shape' anisotropic clustering measurements of \citet{sanchez/etal:2017},
   incorporating information from both redshift-space distortions
   and the baryon acoustic oscillation as our galaxy clustering probe.   We adopt a non-linear
   evolving galaxy bias model, derived from renormalised perturbation theory
   \citep{crocce/scoccimarro:2006, chan/etal:2012}.  
   We maximise the signal-to-noise in our 
   KiDS-BOSS galaxy-galaxy lensing analysis, 
   by including additional overlapping spectroscopy of BOSS-like galaxies from 2dFLenS.

This paper is part of the KiDS-1000 series.  The KiDS-1000 photometry and imaging is presented in \citet{kuijken/etal:2019}.  The core weak lensing data products are presented and validated in \citet[shear measurements,][]{giblin/etal:inprep},  and  \citet[redshift measurements,][]{hildebrandt/etal:inprep}.   \citet{asgari/etal:inprep} conduct the cosmic shear analysis using a range of different two-point statistics, and \citet{joachimi/etal:inprep} detail the methodology behind our `\tttp'  
   analysis, with a particular focus on pipeline validation and accurate covariance matrices.   This paper is organised as follows.   We review the data and provide a concise summary of the findings of the KiDS-1000 series of papers in Section~\ref{sec:data}.   
 We present our joint cosmological constraints in Section~\ref{sec:results}, and conclude in Section~\ref{sec:conc}.  Appendices tabulate the galaxy properties (\ref{app:properties}), the adopted cosmological parameter priors (\ref{app:priors}),  and the cosmological parameter constraints (\ref{app:parameter-constraints}).   They also discuss a series of sensitivity tests (\ref{app:sensitivity}), the redundancy, validation and software review for our pipeline (\ref{app:codereview}), and detail the minor analysis additions that were included after the analysis was formally unblinded (\ref{app:unblinding}).


   











   