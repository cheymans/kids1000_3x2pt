\section{Introduction}
\label{sec:intro}

Observations of the cosmic microwave background (CMB) have delivered high precision
constraints for the cosmological parameters of the flat, cold dark
matter and cosmological constant model of the Universe
\citep[$\Lambda$CDM,][]{planck/etal:2018}.  With only six free
parameters, this flat $\Lambda$CDM model provides an exquisite fit to observations of
the anisotropies in the CMB.    The same model predicts a range of
different observables in the present day Universe, including the
distribution of, and lensing by, large-scale
structures \citep{peebles/1980,bartelmann/schneider:2001,eisenstein/etal:2005}, and the rate
of cosmic expansion \citep{weinberg/1972}.  In broad terms, there
is good agreement between the flat $\Lambda$CDM model cosmological parameters
constrained at the CMB epoch, and those constrained through a variety of
lower redshift probes \citep[see the discussion in][and references
therein]{planck/etal:2018}.   With recent increases in the statistical
precision of the lower redshift probes,
however, statistically significant differences have been found, most
notably a $4.4\sigma$ difference in the value of the Hubble constant
$H_0$ \citep{riess/etal:2019}.  If this difference can not be
attributed to systematic errors in either or both experiments, this result
suggests that the flat $\Lambda$CDM model is incomplete.  Many
extensions have been proposed to reconcile the observed differences between
high and low redshift probes\citep[see for
example][]{poulin/etal:2018,divalentino/etal:2020}.  Each, however, require
an additional component
to the cosmological model that leads even further away from the
standard model of particle physics, that already struggles to motivate
the existence of cold dark matter and a cosmological constant, both core
components of the model.   In these closing phases of the `Stage-III'
surveys, in the run up to the`full-sky' imaging and spectroscopic
cosmology surveys of the 2020's \citep[Euclid, VRO/LSST and DESI,][]{laureijs/etal:2011,lsst/etal:2009,DESI/etal:2016}, the highest
priority is therefore a keen understanding, and mitigation, of systematic errors.

In this analysis we present a multi-probe `same-sky' analysis of the
evolution of large-scale structures, using imaging and spectroscopic surveys.
